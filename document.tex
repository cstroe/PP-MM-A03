\documentclass{article}

\usepackage{graphicx}
\usepackage{multimedia}
\usepackage{amssymb}

%\usepackage{algorithmic}
\usepackage{fancybox}
\usepackage{pseudocode}

\usepackage{pifont}
\usepackage{multirow}
\usepackage{slashbox}
\usepackage{pdfpages}

\title{CS566 Parallel Processing \\ Assignment 03}
\author{Camillo Lugaresi and Cosmin Stroe \vspace{20pt} \\ Department of Computer Science \\
University of Illinois at Chicago}

\date{\today}

\begin{document}

\maketitle
\newpage

\section{Algorithm Details and Formulations}

For this assignment we tackled two main chanllenges.  The first was
experimenting with several implementations of matrix multiplication algorithms. 
In order to do matrix multiplication we implemented Cannon's algorithm and the
Dekel, Nassimi, Sahni (DNS) algorithm.  The second was the optimization of our
LU-decomposition algorithm by eliminating broadcast communication and using
pipelined communication instead.  The reason optimization of the
LU-decomposition algorithm became necessary was that we noticed that it was
taking significantly longer than the matrix multiplication algorithms.  As part
of our optimization we implemented a 1D row partitioning LU-decomposition
algorithm in addition to the 2D partitioning we had implemented for
Assignment 02.  We will be using timing results from both of these algorithms.

In our descriptions, we always refer to the formula $$C = A \times B$$ where
$A$, $B$, and $C$ are $n \times n$ matrices.  Some of the matrix multiplication
algorithms distribute the left and right matrices in a different manner and we
will make the distinction between $A$ and $B$, even though for our problem of
computing $A^k$ they contain identical values.  Our determinant is computed by
running the parallel formulation of the LU decomposition algorithm on the final
matrix, $A^k$.



\subsection{Cannon's Algorithm}

The main benefit of Cannon's algorithm for computing the matrix multiplication
is that it is a \textit{memory space efficient} algorithm, as the storage
requirements per proccessor remain constant as the number of processors is
increased.  The A and B matrices are each distributed in a different order to
fulfill a schedule of operations so that the processors only need to compute a multiplication
between two smaller sub matrices (of size $(n/\sqrt{p})\times(n/\sqrt{p})$ each) at every iteration.  
Because of the way they are distributed, these sub matrices are then shifted
along each dimension of the mesh and another multiplication is done. After
$\sqrt{p}$ iterations of the shift operations, each processor will contain a
submatrix of C, which are then gathered to the root node and prepared for LU
decomposition.  Cannon's algorithm is based on a checkerboard data
decomposition of the output data and uses a 2D mesh toplogy.

\subsubsection{Local memory usage}

Each processor holds a block of the operand matrices A and B and of the result
matrix C. Therefore, $$M_p = O(n^2/p)$$ however, the root node is an exception
since it holds the initial matrix and gathers the final product matrix, so its
local memory usage is $$M_p = O(n^2)$$

\subsubsection{Parallel Computation Time (Local Computations)}

In the analysis of matrix multiplication it is usually assumed that the cost of
multiplication instructions dominates that of addition instructions; this
assumption is reflected in the reality of current computer systems.

For each invocation of the algorithm, each processor performs $\sqrt{p}$ naive
matrix multiplications of submatrices of size $n/\sqrt{p}$. Naive matrix
multiplication performs $O(n^3)$ multiplications, so the number of operations
per processor is $O(\sqrt{p}(n/\sqrt{p})^3)$. Then the total parallel
computation time is: 

\begin{eqnarray*}
T_{{comp}} 	&=& O(t_{{mult}} \cdot p \cdot \sqrt{p} \cdot (n/\sqrt{p})^3) \\ 
			&=& O(t_{{mult}} \cdot n^3)
\end{eqnarray*}

\subsubsection{Parallel Communication Time}

The algorithm involves the following communication steps:
\begin{itemize}
	\item scatter the X matrix into blocks of size $O(n^2/p)$
	\item skew matrix $A$ ($B$); each processor performs a ring\_shift operation on
	the row (column) ring with distance = row (column) number - $\sqrt(p)$ times:
	\item shift matrix $A$ ($B$): ring\_shift with distance 1
	\item gather the product matrix
\end{itemize}

The message size m is always $n^2/p$. In general, we can assume that ts is
dominated by $tw \cdot m$. Scatter on a mesh takes: $$T_{{scatter}} = t_s
\log{p} + t_w \cdot m  \cdot (p-1) = t_s \log{p} + t_w \cdot (n^2/p) \cdot
(p-1)$$. Gather has the same complexity as scatter.

Ring q-shift involves min(q, p-q) neighbor-to-neighbor communications:
$$T_{{shift-q}} = \min{q,p-q} \cdot (t_s + t_w \cdot m)$$

The highest value for the skew shift will be:
$$T_{{skew}} = \sqrt{p}/2 \cdot (t_s + t_w \cdot n^2/p)$$

For all of the sqrt(p) shifts collectively, we have
$$T_{{shift}} = \sqrt{p} \cdot (t_s + t_w \cdot n^2/p)$$

Then the overall communication time is:


\begin{eqnarray*}
T_{{comm}} 	&=& T_{{scatter}} + T_{{skew}} + T_{{shift}} + T_{{gather}} \\
			&=& 2 \cdot T_{{scatter}} + T_{{skew}} + T_{{shift}}  \\
			&=& t_s \cdot 2 \cdot \log{p} + t_w \cdot 2 \cdot (n^2/p)(p-1) + \sqrt{p}/2 \cdot (t_s + t_w \cdot n^2/p) + \sqrt{p} \cdot (t_s + t_w \cdot n^2/p) \\ 
			&=& t_s \cdot 2 \cdot \log{p} + t_w \cdot 2 \cdot (n^2/p)(p-1) + \sqrt{p} \cdot 3/2 \cdot (t_s + t_w \cdot n^2/p) \\ 
			&=& t_s(2 \cdot \log{p} + 3/2 \cdot \sqrt{p}) + t_w (2 \cdot (n^2/p)(p-1) + n^2/p) \\ 
			&=& t_s (2 \log{p} + 3/2 \sqrt{p}) + t_w (n^2/p)(2(p-1) + 1) \\
			&=& t_s (2 \log{p} + 3/2 \sqrt{p}) + t_w (n^2/p)(2p-1) \\
\end{eqnarray*}



\subsubsection{Parallel Run Time}

\begin{eqnarray*}
T_p 	&=& T_{{comp}} + T_{{comm}} \\
		&=& t_{{mult}} \cdot n^3 + t_s (2 \log{p} + 3/2 \sqrt{p}) + t_w (n^2/p)(2p-1) \\
		&=& O(n^3 + \log{p} + \sqrt{p} + n^2 (2p-1)/p) \\
		&=& O(n^3 + \sqrt{p}) \\
\end{eqnarray*}


Since n is always greater than p, we can reduce this to:

$$T_p = O(n^3)$$

\subsection{Cannon's algorithm with local multiplication using Strassen}

\subsubsection{Local memory usage}

In addition to the normal memory usage for Cannon's algorithm, we must consider
the temporary storage used for Strassen. Strassen's algorithm is run locally by
each processor on a square block with size $n/\sqrt{p}$. Let us assume that
$n/\sqrt{p}$ is a power of 2.

Each invocation of Strassen on a matrix of size s allocates storage for
$9*(s/2)^2$ elements (the 7 M matrices and two temporary matrices for computing
intermediate sums and products). Then it recursively invokes Strassen 7 times on
blocks of size $s/2$. Therefore its memory usage is:

\begin{eqnarray*}
f(s) &=& 9/4\cdot s^2 + 7\cdot f(s/2)	\\
&=& 9/4\cdot  s^2 + 7\cdot  9/4\cdot  (s/2)^2 + 7\cdot  7\cdot  f(s/4)	\\
&=& 9/4\cdot  (s^2 + 7/4\cdot  s^2) + 7\cdot  7\cdot  f(s/4)	\\
&=& 9/4\cdot  ( s^2 + 7/4\cdot  s^2 + (7/4)^2\cdot  s^2 ) +7^3\cdot  f(s/2^3) ... 	\\
&=& s^2 \cdot   9/4 \cdot   (7/4 + (7/4)^2 + \dots)	\\
&=& s^2 \cdot   9/4 \cdot   \sum\limits_{i=1}^{log_2 s} (7/4)^i 	\\
&=& s^2 \cdot   9/4 \cdot   log_2(2^{\sum\limits_{i=1}^{log_2 s} (7/4)^i}) 	\\
&=& s^2 \cdot   9/4 \cdot   log_2(\prod\limits_{i=1}^{log_2 s} 2^{(7/4)^i}) 	\\
&=& s^2 \cdot   9/4 \cdot   log_2(\prod\limits_{i=1}^{log_2 s} (7/4)^{2^i}) 	\\
&=& s^2 \cdot   9/4 \cdot   log_2(7/4) \cdot   log_{7/4}(\prod\limits_{i=1}^{log_2 s} (7/4)^{2^i}) 	\\
&=& s^2 \cdot   9/4 \cdot   log_2(7/4) \cdot   (\sum\limits_{i=1}^{log_2 s} log_{7/4}(7/4)^{2^i}) 	\\
&=& s^2 \cdot   9/4 \cdot   log_2(7/4) \cdot   (\sum\limits_{i=1}^{log_2 s} 2^i) 	\\
&=& s^2 \cdot   9/4 \cdot   log_2(7/4) \cdot   (2^{log_2 s+1} - 1) 	\\
&=& s^2 \cdot   9/4 \cdot   log_2(7/4) \cdot   (2\cdot  s - 1) 	\\
&=& s^2 \cdot   (2\cdot  s - 1) \cdot   9/4 \cdot   log_2(7/4) 	\\
&=& (2\cdot  s^3 - s^2) \cdot   9/4 \cdot   log_2(7/4)
\end{eqnarray*}

Hence
$$f(s) = O(s^3)$$

	
$$f(n/\sqrt{p}) = O(n^3/p^{3/2})$$

Thus the space complexity becomes:
$$Mp = O(n^3/p^{3/2} + n^2/p) = O(n^2/p \cdot (n/sqrt(p)+1)) = O(n^3/p^{3/2})$$

For the root node, it becomes:

$$Mp = O(n^3/p^{3/2} + n^2) = O(n^2 \cdot (n/p^(3/2) + 1)) = O(n^3/p^{3/2})$$


\subsubsection{Parallel computation time}

For each invocation of the algorithm, each processor performs sqrt(p) Strassen
matrix multiplications of submatrices of size n/sqrt(p). Strassen matrix
multiplication performs $\approx O(n^{2.807})$ multiplications, so the number of
operations per processor is $O(\sqrt{p}*(n/\sqrt{p})^{2.807})$. Then the total
parallel computation time is:
$$T_{{comp}} = O(t_{{mult}}*p*\sqrt{p}*(n/\sqrt{p})^{2.807})$$
$$O(t_{{mult}}*p^{3/2}*(n/\sqrt{p})^{2.807})$$
$$= O(t_{{mult}}*n^{2.807} * p^{3/2} * p^{-2.807/2}) $$
$$= O(t_{{mult}} * n^{2.807} * p^{(3-2.807)/2}) $$
$$= O(t_{{mult}} * n^{2.807} * p^{0.096})$$

\subsubsection{Parallel communication time}

This is unchanged from the regular Cannon's algorithm:

\begin{eqnarray*}
T_{{comm}} 	&=& T_{{scatter}} + T_{{skew}} + T_{{shift}} + T_{{gather}} \\
			&=& 2 T_{{scatter}} + T_{{skew}} + T_{{shift}} \\
			&=& t_s \cdot 2 \cdot \log{p} + t_w \cdot 2 \cdot (n^2/p)(p-1) + \sqrt{p}/2 \cdot (t_s + t_w \cdot n^2/p) + \sqrt{p}(t_s + t_w \cdot n^2/p) \\
			&=& t_s \cdot 2 \cdot \log{p} + t_w \cdot 2 \cdot (n^2/p)(p-1) + \sqrt{p} \cdot 3/2 \cdot (t_s + t_w \cdot n^2/p) \\
			&=& t_s \cdot (2 \cdot \log{p} + 3/2 \cdot \sqrt{p}) + t_w \cdot (2 \cdot (n^2/p)(p-1) + n^2/p) \\
			&=& t_s \cdot (2 \cdot \log{p} + 3/2 \cdot \sqrt{p}) + t_w \cdot (n^2/p)(2*(p-1) + 1) \\
			&=& t_s \cdot (2 \cdot \log{p} + 3/2 \cdot \sqrt{p}) + t_w \cdot (n^2/p)(2p-1) \\
\end{eqnarray*}

\subsubsection{Parallel run-time}

$$T_p = T_{{comp}} + T_{{comm}} = t_{{mult}} * n^2.807 * p^0.096 + t_s*(2*\log{p} + 3/2*\sqrt{p}) + t_w*(n^2/p)*(2p-1) =$$
$$	= O(n^2.807 * p^0.096 + log(p) + sqrt(p) + n^2*(2p-1)/p) =$$
$$	= O(n^2.807 * p^0.096 + sqrt(p))$$

Since n is always greater than p, we can reduce this to:

$$	T_p = O(n^{2.807} * p^{0.096})$$




\subsection{Dekel, Nassimi, Sahni (DNS) Algorithm}

The DNS algorithm is based on decomposing the intermediate data of the matrix
multiplication algorithm and we chose it because we wanted to explore the effect
of this data decomposition method on our running time. One of the main
differences from Cannon's algorithm is that it uses a 3D mesh topology, and
requires the the number of processors be a perfect cube (i.e., $p = q^3$). Each
processor $P_{i,j,k}$ in the 3D mesh computes the multiplication of the
$A_{i,k}$ and $B_{k,j}$ elements.  In order to make DNS algorithm cost efficient
for $p < n$, each processor receives a submatrix of the A and B matrices, each
of size $(n/q) \times (n/p)$, which it then multiplies using a serial matrix
multiplication algorithm.

The results are then reduced onto the $i-k$ plane and the final $C$ matrix is
gathered at the root processor (rank = 0);

\begin{figure}

\begin{pseudocode}[ruled]{DNS\_Matrix\_Multiplication}{n,k}
\CALL{Create\_3D\_Mesh}{} \\
A \GETS \CALL{AllocateMatrix}{n} \\
B \GETS A \\
i \GETS 1 \\

\WHILE i < k \DO
\BEGIN
	Asub \GETS \CALL{MPI\_Scatter\_on\_I-K}{A} \\
	Bsub \GETS \CALL{MPI\_Scatter\_on\_K-J}{B} \\
	Csub \GETS \CALL{SerialMatrixMultiply}{Asub, Bsub} \\
	
	Cred \GETS \CALL{MPI\_Reduce\_on\_K}{Csub} \\
	C \GETS \CALL{MPI\_Gather\_on\_I-J}{Cred} \\

	B \GETS C \\
\END \vspace{10pt} \\

\RETURN{C}
\end{pseudocode}
\caption{The DNS algorithm with the MPI calls used.}
\end{figure}


\subsubsection{Local memory usage}

Since a cost efficient formulation of the DNS algorithm uses $p < n$, each
processor gets an $(n/q \times n/q)$ submatrix, where $q = \sqrt[3]{p}$, for
each matrix $A$ and $B$.  It also stores the output matrix $C$.  Therefore the
local storage requirements for the DNS algorithm is $3 \cdot n^2 / q^2$ at each
processor.

\subsubsection{Parallel Computation Time (Local Computations)}

Each processor has to serial multiply its submatrices which takes $(n^2/q^2)^3 =
n^5/q^5$ time with the naive matrix multiplication algorithm (if using
Strassen's algorithm this time is $n^{5.614}/q^{5.614}$).  Therefore the
overall time spent for local computations is $$T_{{comp}} = p\cdot
\frac{n^5}{q^5} = q^3 \cdot \frac{n^5}{q^5} = \frac{n^5}{q^2}$$

\subsubsection{Parallel Communication Time}

The parallel communication time comes from the following operations:
\begin{itemize}
  \item Scatter on 2D mesh with each dimension having q processors (done twice).
  \item Broadcast on a ring with q processors (done twice).
  \item Reduce on a ring with q processors.
  \item Gather on 2D mesh with each dimension having q processors.
\end{itemize}

On a 2D mesh the time for a doing a scatter or a gather is
$$T_{{scatter~or~gather},{mesh}} = t_s \log{p} + t_w  m (p-1)$$ and for our
problem $m = n^2$ and $p = q^2$.  The reduction on a ring takes
$$T_{{reduce~or~broadcast},{ring}} = (t_s + t_w\cdot m) \log{p} $$ where for our
problem $m = n^2 / q^2$ and $p = q$.  Therefore our total communication time is:
$$T_{{comm}} = O(\log{n} + n^2 q^2 + \frac{n^2 \log{q}}{q^2} )$$ showing the the
gather and scatter operations will be the most time consuming communication
operations.

\subsubsection{Parallel Run Time}

Using the formula $T_p = T_{{comm}} + T_{{comp}}$ we get that $$T_p = O(n^2
q^2) = O(n^2 p^{2/3})$$

\subsection{LU Decomposition using 2D paritioning (optimized)}

As part of our effort to reduce the running time of the overall algorithm
(matrix multiplication plus computing the determininat) we updated our LU-2D
decomposition algorithm from our implementation in Assignment 02 by removing all
broadcast communications and opting for pipelined communication.  This greatly
reduced the running time of our algorithm.

\subsection{LU Decomposition using 1D partitioning (new)}


The DNS algorithm requires a perfect cube number of processors ($p = q^3$) while
the LU decomposition using a 2D mesh requires a perfect square number of
processors.  Because of the limitations on ARGO, the DNS algorithm can only run
with 8 processes and this is not compatible with our 2D LU decomposition
algorithm.

In order resolve this problem, we also implemented LU Decomposition using 1D
paritioning on a ring topology.  The assignment of the rows were done in a
cyclic fashion in order to balance the workload over all the processors as the
decomposition progresses.

\subsection{Strassen's Matrix Multiplication (serial formulation)}

In order to further improve performance, we implemented the serial version of
Strassen's matrix multiplication algorithm, which we run instead of the naive
matrix multiplication at each node when multiplying the sub matrices distributed
to each node.  The main benefit of Strassen's matrix multiplication is that the
runtime of the algorithm is $\approx O(N^{2.807})$, allowing us to save time
over the standard matrix multiplication with runtime of $O(N^3)$ when dealing
with large matrices .

\section{Parameter Ranges}

Each of the programs were tested with $n \times n$ matrix input sizes of $$n =
\{ 16, 64, 256, 512, 1024, 2048, 4096 \}$$ with powers of $$k=\{ 2, 4, 8,
16 \}$$ and with processors $p$ and cores $c$ ranging from $$(p,c) = \{ (1,4),
(2,2), (4,1), (4,4), (8,2) \}$$

The DNS algorithm can only be run with a perfect cube number of processes, and
on ARGO that limits us to 8 processes.

\section{Results}

\begin{table}[h]
	\centering
\begin{tabular}{|rr|r|r|r|r|r|r|}
\hline
 & \backslashbox{k}{p,c} & (1,1) & (1,4) & (2,2) & (4,1) & (4,4) & (8,2) \\
\hline
\makebox(0,0){\put(0,2.375\normalbaselineskip){\rlap{n}}}
\multirow{2}{*}{16} & 2
& 0.000 & 0.126 & 0.128 & 0.090 & 0.140 & 3.136 \\
& 4
& 0.000 & 0.124 & 0.131 & 0.090 & 0.262 & 0.221 \\
& 8
& 0.001 & 0.126 & 0.130 & 0.093 & 0.185 & 0.187 \\
& 16
& 0.001 & 0.128 & 0.133 & 0.137 & 0.194 & 0.204 \\
\hline
\multirow{2}{*}{64} & 2
& 0.005 & 0.037 & 0.082 & 0.073 & 0.148 & 1.629 \\
& 4
& 0.011 & 0.055 & 0.101 & 0.104 & 0.237 & 0.122 \\
& 8
& 0.022 & 0.062 & 0.104 & 0.114 & 3.120 & 0.258 \\
& 16
& 0.043 & 0.069 & 0.115 & 0.091 & 0.134 & 0.141 \\
\hline
\multirow{2}{*}{256} & 2
& 0.332 & 0.215 & 0.247 & 0.203 & 0.207 & 0.196 \\
& 4
& 1.135 & 0.357 & 0.269 & 0.278 & 0.415 & 0.516 \\
& 8
& 1.804 & 0.687 & 0.618 & 0.648 & 2.561 & 2.692 \\
& 16
& 3.223 & 1.343 & 1.221 & 1.341 & 1.362 & 3.282 \\
\hline
\multirow{2}{*}{512} & 2
& 10.629 & 1.476 & 1.514 & 1.263 & 0.734 & 0.973 \\
& 4
& 29.117 & 3.723 & 2.254 & 3.588 & 1.079 & 1.840 \\
& 8
& 67.642 & 7.645 & 5.010 & 4.736 & 4.092 & 2.876 \\
& 16
& 150.848 & 16.205 & 11.302 & 14.848 & 3.827 & 5.708 \\
\hline
\multirow{2}{*}{1024} & 2
& 90.693 & 57.894 & 25.894 & 23.457 & 5.652 & 4.704 \\
& 4
& 251.303 & 163.092 & 80.998 & 62.297 & 13.904 & 10.239 \\
& 8
& 596.544 & 384.947 & 186.369 & 142.027 & 20.270 & 19.932 \\
& 16
& 1227.563 & 788.244 & 397.010 & 312.980 & 48.695 & 28.922 \\
\hline
\multirow{2}{*}{4096} & 2
& 6682.291 & 3728.372 & 2304.863 & 1682.259 & - & - \\
& 4
& - & 10705.260 & 6674.833 & 4738.603 & - & - \\
& 8
& - & - & - & - & - & - \\
& 16
& - & - & - & - & - & - \\
\hline
\end{tabular}
\caption{Total runtime (in seconds) for the Cannon formulation.}
	\label{tab:cannontotal}
\end{table}
\begin{table}[h]
	\centering
\begin{tabular}{|rr|r|r|r|r|r|r|}
\hline
 & \backslashbox{k}{p,c} & (1,1) & (1,4) & (2,2) & (4,1) & (4,4) & (8,2) \\
\hline
\makebox(0,0){\put(0,2.375\normalbaselineskip){\rlap{n}}}
\multirow{2}{*}{64} & 2
& 0.036 & 0.068 & 0.108 & 0.071 & 0.118 & 0.200 \\
& 4
& 0.107 & 0.104 & 0.127 & 0.093 & 0.169 & 0.115 \\
& 8
& 0.245 & 0.179 & 0.165 & 0.178 & 0.243 & 0.210 \\
& 16
& 0.505 & 0.324 & 0.251 & 0.226 & 0.131 & 0.186 \\
\hline
\multirow{2}{*}{256} & 2
& 1.762 & 0.983 & 0.611 & 0.649 & 0.171 & 0.220 \\
& 4
& 5.115 & 2.814 & 1.579 & 1.595 & 0.328 & 0.261 \\
& 8
& 11.719 & 6.611 & 3.776 & 3.594 & 1.609 & 0.831 \\
& 16
& 24.770 & 13.472 & 7.360 & 8.563 & 1.625 & 1.714 \\
\hline
\multirow{2}{*}{512} & 2
& 12.632 & 6.574 & 4.126 & 3.993 & 0.766 & 3.669 \\
& 4
& 35.885 & 19.064 & 11.312 & 11.025 & 1.108 & - \\
& 8
& 82.428 & 43.501 & 24.208 & 24.836 & 2.145 & - \\
& 16
& 175.750 & 96.441 & 51.518 & 58.017 & 4.226 & - \\
\hline
\multirow{2}{*}{1024} & 2
& 89.222 & 48.935 & 26.795 & 26.862 & 7.889 & - \\
& 4
& 250.350 & 133.086 & 75.469 & 121.289 & 10.205 & - \\
& 8
& 577.792 & 319.440 & 172.052 & 171.236 & - & - \\
& 16
& 1228.597 & 663.503 & 357.386 & 370.693 & - & - \\
\hline
\multirow{2}{*}{4096} & 2
& 4584.392 & 2341.838 & 1285.277 & 1334.404 & - & - \\
& 4
& 12354.114 & 6784.850 & 3575.100 & 3855.360 & - & - \\
& 8
& - & - & 8386.903 & - & - & - \\
& 16
& - & - & - & - & - & - \\
\hline
\end{tabular}
\caption{Total runtime (in seconds) for the Cannon/Strassen formulation.}
	\label{tab:cannon -stotal}
\end{table}
\begin{table}[h]
	\centering
\begin{tabular}{|rr|r|r|r|}
\hline
 & \backslashbox{k}{p,c} & (2,4) & (4,2) & (8,1) \\
\hline
\makebox(0,0){\put(0,2.375\normalbaselineskip){\rlap{n}}}
\multirow{2}{*}{16} & 2
& 0.044 & 0.085 & 0.085 \\
& 4
& 0.045 & 0.085 & 0.044 \\
& 8
& 0.046 & 0.088 & 0.063 \\
& 16
& 0.060 & 0.070 & 0.047 \\
\hline
\multirow{2}{*}{256} & 2
& 0.276 & 0.607 & 1.075 \\
& 4
& 0.509 & 1.576 & 3.014 \\
& 8
& 1.302 & 3.491 & 5.396 \\
& 16
& 2.359 & 6.367 & 10.376 \\
\hline
\multirow{2}{*}{512} & 2
& 1.337 & 1.765 & 3.145 \\
& 4
& 3.906 & 3.219 & 8.211 \\
& 8
& 7.226 & 6.282 & 17.304 \\
& 16
& 14.980 & 14.573 & 39.256 \\
\hline
\multirow{2}{*}{1024} & 2
& 23.208 & 16.034 & 16.589 \\
& 4
& 60.548 & 40.500 & 44.334 \\
& 8
& 136.412 & 86.463 & 89.575 \\
& 16
& 275.903 & 181.148 & 186.428 \\
\hline
\multirow{2}{*}{4096} & 2
& 1383.476 & 810.533 & - \\
& 4
& - & - & - \\
& 8
& - & - & - \\
& 16
& - & - & - \\
\hline
\end{tabular}
\caption{Total runtime (in seconds) for the DNS formulation.}
	\label{tab:dnstotal}
\end{table}
\begin{table}[h]
	\centering
\begin{tabular}{|rr|r|r|r|}
\hline
 & \backslashbox{k}{p,c} & (2,4) & (4,2) & (8,1) \\
\hline
\makebox(0,0){\put(0,2.375\normalbaselineskip){\rlap{n}}}
\multirow{2}{*}{16} & 2
& 0.082 & 0.084 & 0.123 \\
& 4
& 0.083 & 0.044 & 0.083 \\
& 8
& 0.111 & 0.046 & 0.085 \\
& 16
& 0.086 & 0.087 & 0.047 \\
\hline
\multirow{2}{*}{256} & 2
& 0.457 & 1.951 & 1.252 \\
& 4
& 0.611 & 4.628 & 2.558 \\
& 8
& 2.775 & 6.744 & 7.664 \\
& 16
& 3.897 & 7.251 & 12.589 \\
\hline
\multirow{2}{*}{512} & 2
& 2.302 & 2.832 & 2.757 \\
& 4
& 3.879 & 6.812 & 7.645 \\
& 8
& 10.871 & 13.751 & 14.836 \\
& 16
& 17.844 & 25.302 & 33.379 \\
\hline
\multirow{2}{*}{1024} & 2
& 14.736 & 14.632 & 15.081 \\
& 4
& 33.429 & 40.979 & 44.819 \\
& 8
& 62.281 & 89.156 & 89.668 \\
& 16
& 160.716 & 166.998 & - \\
\hline
\multirow{2}{*}{4096} & 2
& - & 1385.929 & - \\
& 4
& - & - & - \\
& 8
& - & - & - \\
& 16
& - & - & - \\
\hline
\end{tabular}
\caption{Total runtime (in seconds) for the DNS/Strassen formulation.}
	\label{tab:dns -stotal}
\end{table}

\begin{table}[h]
	\centering
\begin{tabular}{|rr|r|r|r|r|r|r|}
\hline
 & \backslashbox{k}{p,c} & (1,1) & (1,4) & (2,2) & (4,1) & (4,4) & (8,2) \\
\hline
\makebox(0,0){\put(0,2.375\normalbaselineskip){\rlap{n}}}
\multirow{2}{*}{16} & 2
& 0.000 & 0.000 & 0.001 & 0.001 & 0.043 & 0.045 \\
& 4
& 0.000 & 0.001 & 0.002 & 0.002 & 0.090 & 0.123 \\
& 8
& 0.000 & 0.002 & 0.003 & 0.004 & 0.090 & 0.090 \\
& 16
& 0.001 & 0.003 & 0.006 & 0.008 & 0.099 & 0.104 \\
\hline
\multirow{2}{*}{64} & 2
& 0.003 & 0.001 & 0.002 & 0.002 & 0.004 & 0.004 \\
& 4
& 0.008 & 0.004 & 0.006 & 0.007 & 0.008 & 0.009 \\
& 8
& 0.019 & 0.012 & 0.010 & 0.016 & 0.014 & 0.022 \\
& 16
& 0.041 & 0.020 & 0.021 & 0.033 & 0.024 & 0.029 \\
\hline
\multirow{2}{*}{256} & 2
& 0.221 & 0.083 & 0.054 & 0.056 & 0.030 & 0.029 \\
& 4
& 1.020 & 0.247 & 0.149 & 0.167 & 0.219 & 0.294 \\
& 8
& 1.691 & 0.573 & 0.489 & 0.399 & 0.762 & 0.927 \\
& 16
& 3.110 & 1.230 & 1.091 & 1.189 & 1.125 & 1.502 \\
\hline
\multirow{2}{*}{512} & 2
& 9.689 & 0.928 & 1.080 & 0.772 & 0.312 & 0.276 \\
& 4
& 28.183 & 3.172 & 1.864 & 2.970 & 0.715 & 1.279 \\
& 8
& 66.711 & 7.095 & 4.621 & 4.260 & 2.056 & 2.479 \\
& 16
& 149.922 & 15.641 & 10.872 & 14.347 & 3.478 & 5.297 \\
\hline
\multirow{2}{*}{1024} & 2
& 83.273 & 53.562 & 23.329 & 20.198 & 2.983 & 2.831 \\
& 4
& 243.889 & 158.678 & 78.302 & 59.585 & 12.290 & 5.662 \\
& 8
& 588.994 & 380.529 & 183.592 & 139.308 & 15.574 & 18.707 \\
& 16
& 1220.120 & 783.711 & 394.317 & 310.211 & 43.790 & 27.425 \\
\hline
\multirow{2}{*}{4096} & 2
& 6210.243 & 3477.202 & 2171.188 & 1556.964 & - & - \\
& 4
& - & 10453.918 & 6541.318 & 4613.076 & - & - \\
& 8
& - & - & - & - & - & - \\
& 16
& - & - & - & - & - & - \\
\hline
\end{tabular}
\caption{Multiplication time (in seconds) for the Cannon formulation.}
	\label{tab:cannonmatrix multiplication}
\end{table}
\begin{table}[h]
	\centering
\begin{tabular}{|rr|r|r|r|r|r|r|}
\hline
 & \backslashbox{k}{p,c} & (1,1) & (1,4) & (2,2) & (4,1) & (4,4) & (8,2) \\
\hline
\makebox(0,0){\put(0,2.375\normalbaselineskip){\rlap{n}}}
\multirow{2}{*}{64} & 2
& 0.034 & 0.017 & 0.013 & 0.012 & 0.009 & 0.006 \\
& 4
& 0.105 & 0.055 & 0.033 & 0.035 & 0.023 & 0.008 \\
& 8
& 0.243 & 0.130 & 0.072 & 0.079 & 0.054 & 0.015 \\
& 16
& 0.503 & 0.274 & 0.157 & 0.169 & 0.024 & 0.031 \\
\hline
\multirow{2}{*}{256} & 2
& 1.652 & 0.871 & 0.491 & 0.502 & 0.028 & 0.029 \\
& 4
& 5.005 & 2.703 & 1.453 & 1.443 & 0.074 & 0.097 \\
& 8
& 11.605 & 6.495 & 3.637 & 3.446 & 1.138 & 0.606 \\
& 16
& 24.658 & 13.356 & 7.237 & 8.407 & 1.237 & 1.536 \\
\hline
\multirow{2}{*}{512} & 2
& 11.700 & 6.016 & 3.677 & 3.483 & 0.381 & 0.316 \\
& 4
& 34.958 & 18.518 & 10.872 & 10.526 & 0.714 & - \\
& 8
& 81.496 & 42.953 & 23.721 & 24.360 & 1.742 & - \\
& 16
& 174.824 & 95.898 & 51.082 & 57.544 & 3.793 & - \\
\hline
\multirow{2}{*}{1024} & 2
& 81.757 & 44.546 & 24.230 & 24.184 & 6.080 & - \\
& 4
& 242.965 & 128.713 & 72.928 & 118.600 & 8.581 & - \\
& 8
& 570.359 & 315.016 & 169.502 & 168.558 & - & - \\
& 16
& 1221.206 & 659.062 & 354.858 & 368.032 & - & - \\
\hline
\multirow{2}{*}{4096} & 2
& 4113.942 & 2085.596 & 1150.396 & 1202.956 & - & - \\
& 4
& 11881.555 & 6531.955 & 3439.557 & 3724.244 & - & - \\
& 8
& - & - & 8250.449 & - & - & - \\
& 16
& - & - & - & - & - & - \\
\hline
\end{tabular}
\caption{Multiplication time (in seconds) for the Cannon/Strassen formulation.}
	\label{tab:cannon -smatrix multiplication}
\end{table}
\begin{table}[h]
	\centering
\begin{tabular}{|rr|r|r|r|}
\hline
 & \backslashbox{k}{p,c} & (2,4) & (4,2) & (8,1) \\
\hline
\makebox(0,0){\put(0,2.375\normalbaselineskip){\rlap{n}}}
\multirow{2}{*}{16} & 2
& 0.001 & 0.001 & 0.000 \\
& 4
& 0.001 & 0.001 & 0.001 \\
& 8
& 0.002 & 0.003 & 0.021 \\
& 16
& 0.016 & 0.006 & 0.004 \\
\hline
\multirow{2}{*}{256} & 2
& 0.053 & 0.478 & 0.722 \\
& 4
& 0.387 & 1.221 & 2.612 \\
& 8
& 1.080 & 2.020 & 5.260 \\
& 16
& 2.215 & 4.675 & 9.844 \\
\hline
\multirow{2}{*}{512} & 2
& 0.826 & 1.229 & 2.175 \\
& 4
& 3.392 & 2.713 & 6.461 \\
& 8
& 6.760 & 5.958 & 14.536 \\
& 16
& 14.371 & 14.021 & 35.491 \\
\hline
\multirow{2}{*}{1024} & 2
& 19.811 & 13.719 & 12.432 \\
& 4
& 56.984 & 38.082 & 40.038 \\
& 8
& 133.075 & 83.677 & 86.387 \\
& 16
& 272.513 & 178.654 & 183.061 \\
\hline
\multirow{2}{*}{4096} & 2
& 1218.329 & 714.326 & - \\
& 4
& - & - & - \\
& 8
& - & - & - \\
& 16
& - & - & - \\
\hline
\end{tabular}
\caption{Multiplication time (in seconds) for the DNS formulation.}
	\label{tab:dnsmatrix multiplication}
\end{table}
\begin{table}[h]
	\centering
\begin{tabular}{|rr|r|r|r|}
\hline
 & \backslashbox{k}{p,c} & (2,4) & (4,2) & (8,1) \\
\hline
\makebox(0,0){\put(0,2.375\normalbaselineskip){\rlap{n}}}
\multirow{2}{*}{16} & 2
& 0.001 & 0.001 & 0.000 \\
& 4
& 0.001 & 0.001 & 0.001 \\
& 8
& 0.028 & 0.002 & 0.002 \\
& 16
& 0.004 & 0.005 & 0.004 \\
\hline
\multirow{2}{*}{256} & 2
& 0.299 & 1.767 & 0.937 \\
& 4
& 0.398 & 1.728 & 2.219 \\
& 8
& 2.219 & 3.707 & 7.193 \\
& 16
& 3.731 & 6.825 & 12.065 \\
\hline
\multirow{2}{*}{512} & 2
& 1.250 & 2.453 & 1.881 \\
& 4
& 3.164 & 5.914 & 6.853 \\
& 8
& 10.189 & 12.886 & 13.971 \\
& 16
& 17.194 & 24.684 & 30.270 \\
\hline
\multirow{2}{*}{1024} & 2
& 10.337 & 11.698 & 12.025 \\
& 4
& 29.715 & 36.253 & 41.571 \\
& 8
& 58.128 & 84.742 & 86.160 \\
& 16
& 156.947 & 164.226 & - \\
\hline
\multirow{2}{*}{4096} & 2
& - & 1217.130 & - \\
& 4
& - & - & - \\
& 8
& - & - & - \\
& 16
& - & - & - \\
\hline
\end{tabular}
\caption{Multiplication time (in seconds) for the DNS/Strassen formulation.}
	\label{tab:dns -smatrix multiplication}
\end{table}

\begin{table}[h]
	\centering
\begin{tabular}{|r|r|r|r|r|r|r|}
\hline
\backslashbox{n}{p,c} & (1,1) & (1,4) & (2,2) & (4,1) & (4,4) & (8,2) \\
\hline
16 & 0.000 & 0.122 & 0.124 & 0.095 & 0.086 & 0.085 \\
\hline
64 & 0.002 & 0.040 & 0.084 & 0.073 & 0.134 & 0.130 \\
\hline
256 & 0.104 & 0.108 & 0.117 & 0.122 & 0.167 & 0.147 \\
\hline
512 & 0.899 & 0.510 & 0.359 & 0.445 & 0.271 & 0.329 \\
\hline
1024 & 7.314 & 4.241 & 2.470 & 2.652 & 1.410 & 1.235 \\
\hline
4096 & 469.883 & 248.763 & 130.735 & 121.783 & - & - \\
\hline
\end{tabular}
\caption{LU decomposition time (in seconds) for the Cannon formulation.}
	\label{tab:cannonLU}
\end{table}
\begin{table}[h]
	\centering
\begin{tabular}{|r|r|r|r|r|r|r|}
\hline
\backslashbox{n}{p,c} & (1,1) & (1,4) & (2,2) & (4,1) & (4,4) & (8,2) \\
\hline
64 & 0.002 & 0.046 & 0.090 & 0.063 & 0.125 & 0.150 \\
\hline
256 & 0.102 & 0.101 & 0.111 & 0.132 & 0.182 & 0.154 \\
\hline
512 & 0.894 & 0.506 & 0.396 & 0.422 & 0.320 & 0.267 \\
\hline
1024 & 7.280 & 4.246 & 2.356 & 2.446 & 1.384 & - \\
\hline
4096 & 469.322 & 252.090 & 132.757 & 127.514 & - & - \\
\hline
\end{tabular}
\caption{LU decomposition time (in seconds) for the Cannon/Strassen formulation.}
	\label{tab:cannon -sLU}
\end{table}
\begin{table}[h]
	\centering
\begin{tabular}{|r|r|r|r|}
\hline
\backslashbox{n}{p,c} & (2,4) & (4,2) & (8,1) \\
\hline
16 & 0.043 & 0.079 & 0.053 \\
\hline
256 & 0.172 & 0.907 & 0.352 \\
\hline
512 & 0.497 & 0.461 & 2.298 \\
\hline
1024 & 3.314 & 2.435 & 3.689 \\
\hline
4096 & 163.098 & 95.187 & - \\
\hline
\end{tabular}
\caption{LU decomposition time (in seconds) for the DNS formulation.}
	\label{tab:dnsLU}
\end{table}
\begin{table}[h]
	\centering
\begin{tabular}{|r|r|r|r|}
\hline
\backslashbox{n}{p,c} & (2,4) & (4,2) & (8,1) \\
\hline
16 & 0.082 & 0.063 & 0.082 \\
\hline
256 & 0.269 & 1.633 & 0.408 \\
\hline
512 & 0.728 & 0.736 & 1.200 \\
\hline
1024 & 3.874 & 3.925 & 3.250 \\
\hline
4096 & - & 166.761 & - \\
\hline
\end{tabular}
\caption{LU decomposition time (in seconds) for the DNS/Strassen formulation.}
	\label{tab:dns -sLU}
\end{table}

\begin{table}[h]
	\centering
\begin{tabular}{|rr|r|r|r|r|r|r|r|r|r|r|}
\hline
 & \multirow{2}{*}{\backslashbox{k}{p,c}} & \multicolumn{2}{c|}{(1,1)} & \multicolumn{2}{c|}{(1,4)} & \multicolumn{2}{c|}{(2,2)} & \multicolumn{2}{c|}{(4,4)} & \multicolumn{2}{c|}{(8,2)} \\
n &  & 1d & 2d & 1d & 2d & 1d & 2d & 1d & 2d & 1d & 2d \\
\hline
64 & 2
& 0.005 & - & 0.037 & 0.779 & 0.082 & - & 0.148 & 0.403 & 1.629 & 0.335 \\
\hline
256 & 2
& 0.332 & - & 0.215 & - & 0.247 & 9.697 & 0.207 & - & 0.196 & - \\
\hline
512 & 2
& 10.629 & 16.583 & 1.476 & - & 1.514 & - & 0.734 & - & 0.973 & - \\
\hline
1024 & 2
& 90.693 & 133.654 & 57.894 & - & 25.894 & - & 5.652 & - & 4.704 & - \\
\hline
\end{tabular}
\caption{Total runtime (in seconds) for the Cannon formulation with 1D and 2D LU decomposition.}
	\label{tab:lucompare_total}
\end{table}

\begin{table}[h]
	\centering
\begin{tabular}{|rr|r|r|r|r|r|r|r|r|r|r|}
\hline
 & \multirow{2}{*}{\backslashbox{k}{p,c}} & \multicolumn{2}{c|}{(1,1)} & \multicolumn{2}{c|}{(1,4)} & \multicolumn{2}{c|}{(2,2)} & \multicolumn{2}{c|}{(4,4)} & \multicolumn{2}{c|}{(8,2)} \\
n &  & 1d & 2d & 1d & 2d & 1d & 2d & 1d & 2d & 1d & 2d \\
\hline
64 & 2
& 0.002 & - & 0.033 & 0.775 & 0.077 & - & 0.134 & 0.392 & 0.114 & 0.319 \\
\hline
256 & 2
& 0.102 & - & 0.120 & - & 0.126 & 9.624 & 0.150 & - & 0.135 & - \\
\hline
512 & 2
& 0.905 & 6.830 & 0.506 & - & 0.378 & - & 0.277 & - & 0.408 & - \\
\hline
1024 & 2
& 7.282 & 52.065 & 4.168 & - & 2.379 & - & 1.421 & - & 1.368 & - \\
\hline
\end{tabular}
\caption{LU decomposition time (in seconds) for the Cannon formulation with 1D and 2D LU decomposition.}
	\label{tab:lucompare_LU}
\end{table}



\section{Analysis}

It is difficult to compare Cannon and DNS directly, since the former algorithm requires a number of processors
 that is a square number, while the latter requires a cube. The smallest number of processors that would satisfy both is $2^6 = 64$, but we do not have
 that many processors available on Argo. However, by interpolating the measurements, DNS does not appear to be faster than Cannon's.

Our version of LU decomposition from assignment 2 required a square grid, and thus was not directly compatible with DNS. We implemented a new version with
1D decomposition, which is much faster due to improved pipelining.

Strassen's algorithm barely edges out the naive algorithm in our tests; we would need larger matrices to see a significant difference.


% \begin{figure}
% 	\centering
% 	\includegraphics[width=0.8\textwidth]{images/r-lu}
%     \caption{Runtimes of LU without pivoting using MPI\_Send and MPI\_Recv.}
%     \label{fig:lu_results}
% \end{figure}
% 
% \begin{figure}
% 	\centering
% 	\includegraphics[width=0.8\textwidth]{images/r-lub}
%     \caption{Runtimes of LU without pivoting using MPI\_BCast.}
%     \label{fig:lub_results}
% \end{figure}
% 
% \begin{figure}
% 	\centering
% 	\includegraphics[width=0.8\textwidth]{images/r-lubp}
%     \caption{Runtimes of LU with pivoting using MPI\_BCast.}
%     \label{fig:lubp_results}
% \end{figure}
% 
% \begin{figure}
% 	\centering
% 	\includegraphics[width=0.8\textwidth]{images/r-all}
%     \caption{Comparison of all algorithm with $n=16$.}
%     \label{fig:all_results}
% \end{figure}

%\section{Lessons}



\newpage
~
% \includepdf[pages=-]{code/pdf/cannon-c.pdf}
% \includepdf[pages=-]{code/pdf/cannon-h.pdf}
% \includepdf[pages=-]{code/pdf/dns-c.pdf}
% \includepdf[pages=-]{code/pdf/lu1d-c.pdf}
% \includepdf[pages=-]{code/pdf/lu2d-c.pdf}
% \includepdf[pages=-]{code/pdf/common-c.pdf}
% \includepdf[pages=-]{code/pdf/common-h.pdf}
% \includepdf[pages=-]{code/pdf/Makefile.pdf}

\end{document}
